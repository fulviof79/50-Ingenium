\documentclass[a4paper,12pt]{article} %1
\usepackage[english,italian] {babel} %9
\usepackage[T1]{fontenc}

\usepackage{amsmath}
\usepackage{amssymb}

\usepackage{siunitx}
%\usepackage[output-decimal-marker={,}]{siunitx}

\usepackage{hyperref}
%\setlength{\parindent}{0pt}
%  Utilizzare % per commentare una riga

% Definizione dell'autore dell'articolo
\author{Fulvio Frapolli} %2
%Definizione del titolo
\title{Ingenium \LaTeX}%3
%Definzizione della data
\date{2022-2023} %4 compilare


\begin{document}
    
\maketitle%5

\tableofcontents %8
\newpage %6

\section{La mia prima sezione} %7

Contenuto della mia sezione

\section{Scrittura con elementi matematici}

Necessita i pacchetti \verb |amsmath| e \verb|amssymb|

Posso inserire scrittura matematica in una riga $x^2-\sqrt{3}=\frac{1}{3-x^4}$

Rendere le frazioni più visibili  $\alpha \cdot  x^2-\sqrt{3}=\dfrac{1}{3-x^4}$

Nuova riga e centrato 

\[
\frac{1}{a-\frac{1}{a+1}}    
\]


\[
\underbrace{1+2+\dots+n}_%
{=\frac{n(n+1)}{2}}
+(n+1)
\]


\section{Unità di misura}

Caricare il pacchetto \verb|siunitx|

\qty{23.4}{kg.m.s^{-2}} \\
$r=\qty{0.8768(11)e-15}{m}$ \\
\unit{\joule\per\mole\per\kelvin}\\
\unit{j.mol^{-1}.K^{-1}}


\qty{100}{\celsius} \\
\ang{1;2;3}

\end{document}