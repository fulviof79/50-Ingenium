\documentclass[a4paper,12pt]{scrartcl}

\usepackage[italian]{babel}
\usepackage[T1]{fontenc}	% ascii 8 bit
\usepackage{lmodern}		% latin modern
\usepackage{siunitx}

\title{Progetto 2}
\author{Reto Ambrosini}
\date{\today}

\begin{document}
	
	\maketitle
	
\newpage

	\tableofcontents

\newpage

\section{Prima sezione}

Per scrivere un'equazione nel testo $ x=\frac{2a+3}{a-1} $ utilizzo il segno \$.
Invece per scrivere un'equazione centrata si utilizza

\[ x=\frac{2a+3}{a-1} \]

oppure

$$ \sin(x)=\frac{2a+3}{a-1} $$

oppure ancora

\begin{equation}
	\sin(x)=\frac{2a+3}{a-1} 
\end{equation}

\vspace{\stretch{1}}

\section[Due]{Seconda sezione}
	
Qui viene il testo della seconda sezione.
Qui viene il testo della seconda sezione.
Qui viene il testo della seconda sezione.
Qui viene il testo della seconda sezione.

\vspace{2mm}
Qui viene il testo della seconda sezione.
Qui viene il testo della seconda sezione.
Qui viene il testo della seconda sezione.

\vspace{\stretch{1}}

\section*{Allegati}
\addcontentsline{toc}{section}{Allegati}

Qui viene il testo della seconda sezione.

\vspace{\stretch{2}}

\end{document}